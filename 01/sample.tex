\documentclass[a4j]{jarticle}

% プリアンブル部

\title{はじめての\LaTeX}
\date{\today}
\author{自分の名前}

\begin{document}

  \maketitle
  
  \LaTeX は最高級の自動組版システムである.
  \LaTeX では, 数式を美しく表示できる他, 章番号, 節番号などを自動的につけることができる.
  また, 目次, 索引, 文献リストも自動的に作ることができる.
  なんと便利なのだろう.

  \LaTeX では, \LaTeX ファイル内で1行改行しても, PDF ファイル内では改行されない.\\
  改行をするには, バックスラッシュを改行したい場所に2つ入力するか,

  1行以上の空白行を入力する必要がある.

  \section{定積分の定義}
  関数$y=f(x)$は, 閉区間$[a, b]$で定義され, $f(x)\ge0$であるとする.
  この区間を$n$個の小区間に分け, 小区間$[x_{k-1}, x_k]$の幅を$\Delta x_k = x_k - x_{k-1}$, 小区間の最大値を$|\Delta| = \max{\Delta x_k}$ とするとき, $f(x)$の定積分は
  \[
    \int_a^b f(x)dx = \lim_{|\Delta| \to 0} \sum_{n}^{k=1}f(x_k)\Delta x_k
  \]  
  と定義される\cite{math2022}.

  \begin{thebibliography}{99}
    \bibitem{math2022}
    数学太郎, 微分積分学の基礎, Hoge出版, 2022.
  \end{thebibliography}

\end{document}